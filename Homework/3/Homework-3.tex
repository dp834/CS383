\title{CS 383 - Machine Learning}
\author{
        Assignment 3 - Dimensionality Reduction\\
}
\date{}
\documentclass[12pt]{article}
\usepackage[margin=0.7in]{geometry}
\usepackage{graphicx}
\usepackage{float}
\usepackage{comment}
\usepackage{amsmath}
\usepackage{listings}

\begin{document}
\maketitle


\section{Theory Questions}
\begin{enumerate}

\item Consider the following data:\\
\begin{center}
$
 \begin{bmatrix}
    -2 & 1\\
    -5 & -4\\
    -3 & 1\\
    0 & 3\\
    -8 & 11\\
    -2 & 5\\
    1 & 0\\
    5 & -1\\
    -1 & -3\\
    6 & 1\\
\end{bmatrix}
$
\end{center}
    \begin{enumerate}
    \item Find the principle components of the data (you must show the math, including how you compute the eigenvectors and eigenvalues).
    Make sure you standardize the data first and that your principle components are normalized to be unit length.
    As for the amount of detail needed in your work imagine that you were working on paper with a basic calculator.
    Show me whatever you would be writing on that paper.
    $$ mean_0 = \frac{-2 -5 -3 0 -8 -2 1 5 -1 6}{10} = -.9 \quad mean_1 = \frac{1 -4 1 3 11 5 0 -1 -3 1}{10} = 1.4 $$
    $$ stdev_0 = \sqrt{\frac{\sum(x_{0i} - mean_0)}{10}} = 4.011  \quad stdev_1 = 4.055 $$
\begin{center}
$
A = \begin{bmatrix}
-0.27422981 & -0.09865272 \\
-1.02212929 & -1.33181178 \\
-0.52352964 & -0.09865272 \\
 0.22436984 &  0.3946109  \\
-1.77002878 &  2.36766539 \\
-0.27422981 &  0.88787452 \\
 0.47366967 &-0.34528454  \\
 1.47086898 & -0.59191635 \\
-0.02492998 & -1.08517997 \\
 1.72016881 & -0.09865272
\end{bmatrix}
$
\end{center}
$$\text{Covariance:} C = \frac{1}{N-1}A^TA =
\begin{bmatrix}
     1.111 & -0.454 \\
    -0.454 &  1.111
\end{bmatrix}$$
$$\text{Eigenvalues:} |C - \lambda I| = 0 \implies (1.111 - \lambda)(1.111 - \lambda) - (-0.454)(-0.454)$$
$$ \implies \lambda^2 - 2.222\lambda + 1.111^2 - 0.454^2 = \lambda^2 - 2.222\lambda + 1.028205$$
$$ \lambda_{1,2} = \frac{- (-2.222) \pm \sqrt{(-2.222)^2 - 4(1)(1.028205)}}{2(1)} = 1.111 \pm \sqrt{1.111^2 -1.028205}$$
$$ \lambda_1 = 1.5615 \quad \lambda_2 = 0.6585 $$
$$ \text{vec}_1 =
\begin{bmatrix}
     1.111 & -0.454 \\
    -0.454 &  1.111
\end{bmatrix}
x = 1.5615 x
\implies
\begin{array}{c}
1.111x_0 - 0.454x_1 = 1.5615x_0 \\
-0.454x_0 + 1.111x_1 = 1.5615x_1
\end{array}
$$

$$
\implies
\begin{array}{c}
-0.454x_0 -0.454x_1 = 0 \\
-0.454x_0 + -0.454x_1 = 0
\end{array}
\implies x_0 = x_1 \implies v_1 =
\begin{bmatrix}
-1 \\
1
\end{bmatrix}
$$
$$
\frac{v_1}{|v_1|} = \frac{1}{\sqrt{1^2+1^2}} \begin{bmatrix} -1 \\ 1 \end{bmatrix}
= \begin{bmatrix} -\frac{1}{\sqrt{2}} \\ \frac{1}{\sqrt{2}} \end{bmatrix}
\approx \begin{bmatrix} -0.7071 \\ 0.7071 \end{bmatrix}
$$

$$ \text{vec}_2 =
\begin{bmatrix}
     1.111 & -0.454 \\
    -0.454 &  1.111
\end{bmatrix}
x = 0.6585 x
\implies
\begin{array}{c}
1.111x_0 - 0.454x_1 = 0.6585x_0\\
-0.454x_0 + 1.111x_1 = 0.6585x_1
\end{array}
$$

$$
\implies
\begin{array}{c}
0.454x_0 -0.454x_1 = 0 \\
-0.454x_0 + 0.454x_1 = 0
\end{array}
\implies -x_0 = x_1 \implies v_1 =
\begin{bmatrix}
1 \\
1
\end{bmatrix}
$$
$$
\frac{v_2}{|v_2|} = \frac{1}{\sqrt{1^2+1^2}} \begin{bmatrix} 1 \\ 1 \end{bmatrix}
= \begin{bmatrix} \frac{1}{\sqrt{2}} \\ \frac{1}{\sqrt{2}} \end{bmatrix}
\approx \begin{bmatrix} 0.7071 \\ 0.7071 \end{bmatrix}
$$
$$ \lambda_1 = 1.5615 \quad v_{\lambda_1} = \begin{bmatrix} -0.7071 \\ 0.7071 \end{bmatrix} $$
$$ \lambda_2 = 0.6585 \quad v_{\lambda_2} = \begin{bmatrix}  0.7071 \\ 0.7071 \end{bmatrix} $$
    \item Project the data onto the principle component corresponding to the largest eigenvalue found in the previous part (3pts).
        \newline
        Largest eigenvalue = $\lambda_1 = 1.5615$ with the corresponding vector $v_{\lambda_1}\begin{bmatrix} -0.7071 \\ 0.7071 \end{bmatrix}$.
        To project we just need to do one matrix multiplication
\begin{center}
$$
 \begin{bmatrix}
-0.27422981 & -0.09865272 \\
-1.02212929 & -1.33181178 \\
-0.52352964 & -0.09865272 \\
 0.22436984 &  0.3946109  \\
-1.77002878 &  2.36766539 \\
-0.27422981 &  0.88787452 \\
 0.47366967 &-0.34528454  \\
 1.47086898 & -0.59191635 \\
-0.02492998 & -1.08517997 \\
 1.72016881 & -0.09865272
\end{bmatrix}
\begin{bmatrix}
-0.7071 \\
0.7071
\end{bmatrix}
=
\begin{bmatrix}
 0.12415175 \\
-0.21897859 \\
 0.30043335 \\
 0.12037861 \\
 2.92579161 \\
 0.82173185 \\
-0.57908808 \\
-1.45860949 \\
-0.74970996 \\
-1.28610104
\end{bmatrix}
$$
\end{center}
    \end{enumerate}
\end{enumerate}


\newpage
\section{Dimensionality Reduction via PCA}
\begin{itemize}
    \item Built in, sklearn knn where k=1, Score: $0.23$
    \item My knn where k=1, Score: $0.23$
    \item My knn where k=1 with 100 principle components, Score: $0.25$
    \item My knn where k=1 with 100 principle components and whitened, Score: $0.30$
\end{itemize}
\begin{figure}[h]
\centering
\includegraphics[scale=.4]{images/Q2-PC1-vs-PC2.png}
\caption{Visualization of reduction to 2D using PCA.}
\end{figure}

\begin{figure}[h]
\centering
\includegraphics[scale=.4]{images/Q2-PC1-vs-PC2-Whitened.png}
\caption{Visualization of reduction to 2D using PCA and whitened.}
\end{figure}

\newpage
\section{Eigenfaces}
\begin{figure}[h!]
    \centering
    \includegraphics[scale=.7]{images/Q3-PC1.png}
    \caption{The first Principle component}
\end{figure}

\begin{figure}[h!]
    \centering
    \includegraphics[scale=.7]{images/Q3-PC1-PC2-min-max.png}
    \caption{The max and min photos for Principle components 1 and 2}
\end{figure}
We can see from the previous image PC1 appears to focus on the face contrasting to the background as well as basic eye nose and mouth placement.
The min for PC1 has a bright face with a darker background behind him while the max for PC1 has the opposite, a dark center with a brighter background.
\newline
Meanwhile it appears the PC2 deals with the left and right side of a face being lit in comparison to the other side.
The min for PC2 has a bright left side and a dark right side while the max for PC2 again has the opposite, a dark left side and a bright right side.

\begin{figure}[h!]
    \centering
    \includegraphics[scale=.7]{images/Q3-PC1-reconstruction.png}
    \caption{Reconstruction of the first image in the test data using just the first principal component.}
\end{figure}

\begin{figure}[h!]
    \centering
    \includegraphics[scale=.7]{images/Q3-image-95-reconstruction.png}
    \caption{Reconstruction of the first image in the test data keeping 95\% of the variance. \newline
             This image used 188 principal components to capture 95\% of the variance in the image.}
\end{figure}

\end{document}

